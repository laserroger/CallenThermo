\chapter{其它形式与勒让德变换}
\label{chap5}

\section{能量最小原理}
\label{sec5.1}

在之前的章节中,我们已经得到了熵最大原理的最显著和直接的一些推论。
进一步的推论会导出很多其它有用的基本结论。
不过现在我们重新考虑理论的形式并留意到用一些等价的数学形式可以重新构造出相同的内容,会对促进这些问题的发展更加有用。
这些其它的形式每一个都会在一些特殊类型的问题中显得特别方便面,
而热力学计算的艺术大多数情况都体现在选取对于给定的问题最适用的某种理论形式上。
在恰当的形势下热力学问题会变得极为简单,
另一方面则是在不恰当的下问题会变得极为复杂。

力学中也会出现多种等价的形式——牛顿形式,拉格朗日形式,哈密顿形式是完全等价的。
同样这时也会某一问题用拉格朗日形式处理会比用牛顿形式更加容易的请况,反之亦然。
但是不同形式导致的便利程度的不同在热力学中会表现得极为显著。
正因如此,{\it 等价表述间的变换的普遍性理论被视为热力学理论的一个基本方面}。

实际上我们已经考虑过了两种等价的表述——能量表述和熵表述。
但基本的极值原理则只在熵表述中被构造出来。
如果这两种表述在理论中的地位是相同的,我们就必须找出能量表述中与熵最大原理类似的极值原理。
这实际上应该是一个与熵最大原理等价的极值原理,并且可以用能量最小原理取代。
熵最大原理告诉我们对于一个给定总能量的平衡态熵会取最大值。
同理,能量最小原理则告诉我们对于给定总熵的平衡态,能量会取最小值。

图\ref{fig5.1}展示了在\ref{sec4.1}节中讨论的复合系统的味型空间的刨面图。
标记为$U$和$S$的轴对应着复合系统的总能量和总熵,而标记为$X_J^{(1)}$的轴则对应着第一个子系统的某个广延参量。
其它没有在图中显示的轴还包括$U^{(1)}$,$X_J$以及其它的$X_K^{(1)}$和$X_K$。

复合系统的总能量是一个由封闭条件决定的常数。
这个封闭条件的几何表示是系统的态处于图\ref{fig5.1}中$U=U_0$的平面上。
系统的基本方程表示为图中的曲面,因此表示系统状态的点一定位于曲面和平面相交得到的曲线上。
如果参数$X_J^{(1)}$不受约束,那么平衡态就是曲线上使得熵最大的态,也就是图\ref{fig5.1}中标记为A的态。

将平衡态A当做给定熵的时候能量最小态的另一种表述显示在图\ref{fig5.2}中。
经过平衡态点A的平面$S=S_0$与基本曲面相交定义了一条曲线。
这条曲线包含了一族熵为常数的态,{\it 而平衡态A则是曲线上使得能量最小的态}。

熵最大和能量最小原理的等价性明确依赖于基本曲面的几何形状这个性质,如同图\ref{fig5.1}和\ref{fig5.2}所示是普遍的。
如同\ref{sec4.1}节中讨论过的那样,图中显示的曲面的形状是由$\partial S/\partial U>0$以及U是关于S的单值连续函数这两个假设决定的;
因此这两个解析的假设是两个原理等价的隐含条件。

总结一下,尽管我们尚未证明,但看起来以下两个原理是等价的:

{\bf 熵最大原理。}{\it 在总的内部能量确定时,任何不受约束的参数在平衡时的取值都使得熵最大}。

{\bf 能量最小原理。}{\it 在总的熵确定时,任何不受约束的参数在平衡时的取值都使得能量最小}。

两个极值准则的等价性的证明即可以用物理论证的方式阐述,也可以用数学证明的方式阐述。
我们首先考虑物理讨论,来论证如果能量{\it 不}是最小值的话那么熵就可以不是最大值,并且反之亦然。

接下来,假设系统处于平衡态但能量并{\it 不}是在给定的熵下可能的最小的值。
我们就可以在保持熵为常数的同时从系统中提取能量(用功的形式),并且我们随后就可以用热量的形式把这部分能量返回
这样系统的熵会增加($\bar{d}Q=TdS$),并且系统会恢复到它初始的能量但是熵增加了。
这跟初始的平衡态是熵最大的态是矛盾的!
因此我们不得不推断最初的平衡态必须具有给定的熵下最小的能量。。

相反的推论,也就是最小能量要求最大的熵,可以用类似的方式构造(见问题 5.1-1).

在更形式化的证明中,我们假设熵最大原理
\begin{equation}
\label{equ5.1}
\left(\frac{\partial S}{\partial X}\right)_U=0
~\text{and}~
\left(\frac{\partial^2 S}{\partial X^2}\right)_U<0
\end{equation}
在此为了简便,我们将$X_J^{(1)}$写作$X_J$,这暗示着其它的$X$将保持为常数。
同时为了简便,我们暂时将一阶导数$(\partial U/\partial U)_S$记为$P$。
于是根据(根据附录\ref{appendix.A}中的公式\eqref{equA.22})

\begin{equation}
\label{equ5.2}
P\equiv\left(\frac{\partial U}{\partial X}\right)_S
=-\frac{\left(\frac{\partial S}{\partial X}\right)_U}{\left(\frac{\partial S}{\partial U}\right)_X}
=-T\left(\frac{\partial S}{\partial X}\right)_U=0
\end{equation}

我们得到$U$是极值。
为了分清这个极值是极大极小还是一个拐点,我们必须研究二阶导数$(\partial^2U/\partial X^2)_S\equiv(\partial P/\partial X)_S$的符号。
但是如果将$P$当做$U$和$X$的函数我们有
\begin{equation}
\label{equ5.3}
\left(\frac{\partial^2 U}{\partial X^2}\right)_S
=\left(\frac{\partial P}{\partial X}\right)_S
=\left(\frac{\partial P}{\partial U}\right)_X
\left(\frac{\partial U}{\partial X}\right)_S
+\left(\frac{\partial P}{\partial X}\right)_U
=\left(\frac{\partial P}{\partial U}\right)_XP
+\left(\frac{\partial P}{\partial X}\right)_U
\end{equation}
\begin{equation}
\label{equ5.4}
=\left(\frac{\partial P}{\partial X}\right)_U.~\text{若}P=0
\end{equation}
\begin{equation}
\label{equ5.5}
=\frac{\partial}{\partial X}
\left[-\frac{\left(\frac{\partial S}{\partial X}\right)_U}{\left(\frac{\partial S}{\partial U}\right)_X}\right]_U
\end{equation}
\begin{equation}
\label{equ5.6}
=-\frac{\frac{\partial^2S}{\partial X^2}}
{\frac{\partial S}{\partial U}}+\frac{\partial S}{\partial X}
\frac{\frac{\partial^2S}{\partial X\partial U}}{\left(\frac{\partial S}{\partial U}\right)^2}
\end{equation}
\begin{equation}
\label{equ5.7}
=-T\frac{\partial^2S}{\partial X^2}>0
~\text{若}\frac{\partial S}{\partial X}=0
\end{equation}
因此$U$是一个极小值。
反向的论证形式上是一样的。

如同已经指出的,两个极值条件描述的情形是完全一样的这一事实类似于几何中的等周问题。
也就是一个圆既可以描述为给定周长下面积最大的二维图形,也可以描述为给定面积下周长最小的二维图形。


两个可供选择的用于描述圆的极值条件是完全等价的,而且每一个都适用于所有的圆。
不过他们给出两种不同的生成圆的方法。
假设我们有一个正方形并且想使它连续变形为一个圆。
我们可以保持它的面积是常数并且允许约束的曲线像橡皮筋一样收缩。
这样我们得到了作为给定面积下周长最小图形的一个圆。
等价地,我们也可以保持正方形的周长不变并允许面积增加,这样我们就得到了作为给定周长下面积最大图形的一个(不同的)圆。
尽管如此,在得到这些圆之后,{\it 每一个圆都同时满足自身取值下面积和周长的极值条件}。

关于热力学系统的物理情形跟上面描述的几何的情形是非常类似的。
也就是说,任何一个平衡态既可以描述为给定能量下熵最大的态,也可以描述为给定熵下能量最小的态。
但是这两个条件却给出了两个不同的达到平衡态的方法。
作为这两种接近平衡的方法的一个特定的例子,考虑一个初始位于一个封闭圆柱的某个位置的活塞。
我们的兴趣在于不约束活塞位置的同时使系统达到平衡。
我们可以简单地移去约束来使它自发达到平衡;此时由于封闭条件能量会保持为常数同时熵会增加。
这个过程就是熵最大原理要求的过程。
等价地,我们可以让活塞非常缓慢地移动,可逆地对外做功直到它到达使两边压强相等的位置。
在这个过程中能量被从系统中提取出来,但是它的熵保持为常数(这个过程是可逆的且没有热流)。
这个过程就是能量最小原理要求的过程。
我们想强调的关键事实是,{\it 与通过二者之一或是其它的某种过程达到平衡态的具体过程无关,通过各种方式达到的最终的平衡态都满足两种极值条件}

最后我们通过以之替代熵最大原理来解决\ref{sec2.4}节中的热平衡问题来阐明能量最小原理。
我们考虑一个内部具有一个导热的刚性隔离墙的封闭的复合系统。
热量可以在两个子系统间自由流动,因此我们想找到它的平衡态。
在能量表述下的基本方程是
\begin{equation}
\label{equ5.8}
U=U^{(1)}(S^{(1)},V^{(1)},N^{(1)}_1,\ldots)+U^{(2)}(S^{(2)},V^{(2)},N^{(2)}_1,\ldots)
\end{equation}
所有的关于体积和摩尔数的参数都是已知的常数。
需要被计算的变量是$S^{(1)}$和$S^{(2)}$。
现在,忽略系统封闭导致的总能量不变的事实,{\it 如果}能量的改变被允许的话,平衡态可以被表示为使得能量最小的态。
总能量关于两个系统中虚拟热流的虚拟变化是
\begin{equation}
\label{equ5.9}
dU=T^{(1)}dS^{(1)}+T^{(2)}dS^{(2)}
\end{equation}
能量最小条件也就是$dU=0$,在服从总熵不变的条件时:
\begin{equation}
\label{equ5.10}
S^{(1)}+S^{(2)}=\text{constant}
\end{equation}
会有
\begin{equation}
\label{equ5.11}
dU=(T^{(1)}-T^{(2)})dS^{(1)}=0
\end{equation}
于是我们可以推出
\begin{equation}
\label{equ5.12}
T^{(1)}=T^{(2)}
\end{equation}

从而能量最小原理对我们给出了与我们前面用熵最大原理得到的相同的热平衡条件。

方程\eqref{equ5.12}是一个关于$S^{(1)}$和$S^{(2)}$的方程。
在总能量$U$已知,因此仅有的两个未知量是$S^{(1)}$和$S^{(2)}$的情况下,第二个方程选为方程\eqref{equ5.8}最为方便。
理论上方程\eqref{equ5.8}和\eqref{equ5.12}会给出这个问题的精确解。

用完全一样的思路,可以发现一个内部带有可动绝热墙的封闭系统的平衡条件是压强相等。
这个结论在能量表述中是非常直接的,但是如同在\ref{sec2.7}的最后一段看到的,在熵表述中相对更加微妙。



